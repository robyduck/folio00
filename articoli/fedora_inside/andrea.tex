\twocolumn[
\begin{center}
\title{\color[cmyk]{1, 0.57, 0, 0.38}{\Huge\bfseries Fedora Inside\\}} % definisco il titolo dell'articolo
\author{\scriptsize Gabriele Trombini (mailga@fedoraonline.it)} % definisco l'autore e altre informazioni
\date{}
\end{center}
{\color[cmyk]{1, 0.46, 0, 0}\LARGE Una collaborazione speciale - Andrea Veri}\\
\maketitle
\normalsize
\doublespacing
\hfill
]
\onehalfspacing
\lettrine[lines=1, loversize=0.1, lraise=0.1]{\color[cmyk]{0.5, 0, 1, 0}\bfseries F}{}in da questa prima uscita e con grande piacere, possiamo avvalerci della collaborazione di un amico che opera nell'infrastruttura del Fedora Project e in posizioni prestigiose all'interno del panorama GNU/Linux.\\ 

Visto la natura delle argomentazioni che Andrea tratterà all'interno di Folio (principalmente sicurezza e sistemi, talvolta spiegherà dinamiche viste dall'interno del progetto) la presenza dei suoi contributi, ovviamente, non potrà avere il carattere della continuità.\\

Andrea Veri è un ventiduenne studente della facoltà di Giurisprudenza dell'Università di Udine.\\

Da sempre appassionato di informatica, iniziò nel 2006 a contribuire nei campi della documentazione e del packaging di applicazioni in Ubuntu per poi spostarsi successivamente in Debian, dove ottenne il titolo di sviluppatore ufficiale con i relativi accessi all'archivio principale della distribuzione.\\

Dal 2009, iniziò la sua collaborazione con il progetto GNOME nel quale ricopre alcune posizioni di carattere infrastrutturale, nello specifico, si occupa di amministrare e gestire il server LDAP di GNOME e presiede la GNOME Membership \& Elections Committee, la quale si occupa di valutare, accettare o rifiutare le richieste di appartenenza alla Fondazione GNOME e non solo: tale commissione è, inoltre, incaricata di organizzare le elezioni della Board of Directors a Giugno prima del GUADEC.\\

Dal 2010, Andrea si sposta, pur mantenendo attive le altre sue collaborazioni con Debian e GNOME, in Fedora.\\

I primi gruppi di lavoro furono relativi alle traduzioni in lingua italiana, al gruppo Insight ed a quello Infrastructure.\\

Seguì poi la creazione del sub-planet JustFedora e la successiva candidatura alla Board di Fedora.\\

Al momento, le sue attività si concentrano maggiormente nel packaging RPM e nella amministrazione di sistema di alcune macchine utilizzate dall'infrastruttura di Fedora.\\