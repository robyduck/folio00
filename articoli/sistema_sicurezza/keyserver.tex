\twocolumn[
\begin{center}
\title{\color[cmyk]{1, 0.57, 0, 0.38}{\Huge\bfseries Keyserver e generazione di chiavi GPG \\}} % definisco il titolo dell'articolo
\author{\scriptsize Andrea Veri (averi@fedoraproject.org)} % definisco l'autore e altre informazioni
\date{}
\end{center}
{\color[cmyk]{1, 0.46, 0, 0}\LARGE Vediamo come usare il keyserver Fedora per la generazione di chiavi GPG}\\
\maketitle
\normalsize
\doublespacing
\hfill
]
\onehalfspacing
\lettrine[lines=1, loversize=0.1, lraise=0.1]{\color[cmyk]{0.5, 0, 1, 0}\bfseries D}{}a pochi giorni ha debuttato in Fedora un interessante servizio denominato "Keys" (raggiungibile all'indirizzo http: //keys.fedoraproject.org), si tratta, come si può ben intendere dal nome usato, di un server adibito allo storage di chiavi OpenPGP.\\

Questa tipologia di servizio permette l'inivio e la ricezione di specifiche chiavi di cifratura PGP, le quali vengono selezionate dall'utente a seconda della sua necessità di identificare o di firmare la chiave di un altro soggetto.\\

Ma come funziona esattamente GPG? \\

Il meccanismo risulta essere alquanto semplice ed è, inoltre, similare a quello utilizzato dalla firma digitale, forma di autenticazione dei documenti telematici che oramai sta prendendo piede anche in Italia.\\

GPG, quindi, cifra un determinato messaggio utilizzando una coppia di chiavi (pubblica e privata) generate dall'utente. Il soggetto che riceverà tale messaggio cifrato non dovrà fare altro che verificare l'autenticità della firma e verificarne l'attendibilità. Onde evitare la divulgazione di chiavi false o fittizie e favorire l'associazione chiave pubblica - u\-tente, si è giunti a dare particolare rilevanza al cosiddetto "Web of Trust" (o più comunemente "WoT").\\

Tale concetto è di facile comprensione: si pensi ad un determinato utente che durante un meeting incontra altri individui e dopo aver provveduto ad identificarli tramite un documento d'identità e tramite la verifica della loro impronta digitale (o più comunemente "fingerprint"), firma la loro chiave PGP con la propria, certificandone l'autenticità della provenienza.\\

Questa semplice ma efficace operazione aiuta a rafforzare la rete di fiducia e di conseguenza la stessa credibilità dell'utente stesso.\\

Detto ciò seguirà una brevissima guida che coprirà principalmente tre tematiche:\\
\definecolor{shadecolor}{cmyk}{0, 0, 0, 0.8}
   
\begin{enumerate}
\item la creazione di una chiave PGP 
\item la firma di una chiave PGP altrui
\item la cifratura di un file decifrabile solamente da una determinata chiave
\end{enumerate}

{\bfseries\centering Punto 1. Creazione di una chiave PGP.}\\

\begin{itemize}
\item 1a. Installiamo il software di riferimento.\\
\begin{shaded}
{\color[cmyk]{0, 0, 0, 0}\textdollar\ sudo yum install gnupg}\\
\end{shaded}
\item 1b. Apriamo un terminale e creiamo la nostra chiave di cifratura.\\
\begin{shaded}
{\color[cmyk]{0, 0, 0, 0}\textdollar\ gpg --gen-key}\\
\end{shaded}
Le impostazioni di default, ovvero quelle che otterrete con un semplice 'Invio' ad ogni richiesta di input da parte di GPG, sono solitamente quelle consigliate ad un utente normale.\\
È però consigliabile impostare una data di scadenza alla chiave nel qual caso perdeste o cadesse in mani sbagliate la vostra chiave privata.\\
\item 1c. Appena generata la chiave, inviatela al Keyserver di Fedora, rendondola così pubblicamente disponibile.\\
\begin{shaded}
{\color[cmyk]{0, 0, 0, 0}\textdollar\ gpg --send-keys KEYID --keyserver hkp://keys.fedoraproject.org}\\
\end{shaded}
Per visualizzare il vostro KEYID, vi basterà digitare il seguente comando:
\begin{shaded}
{\color[cmyk]{0, 0, 0, 0}\textdollar\ gpg --list-secret-keys}\\
\end{shaded}
Il KEYID sarà la prima linea numerica che vedrete, ad esempio: sec   4096R/93ECD6F8 2012-01-31\\
\end{itemize}

{\bfseries\centering  Punto 2. Firma di una chiave PGP altrui.}\\

\begin{itemize}
 \item 2a. Otteniamo la chiave che vogliamo firmare dal Keyserver di riferimento.\\
\begin{shaded}
{\color[cmyk]{0, 0, 0, 0}\textdollar\ gpg --recv-key KEYID --keyserver hkp://keys.fedoraproject.org}\\
\end{shaded}
\item 2b. Firmiamo la chiave.\\
\begin{shaded}
{\color[cmyk]{0, 0, 0, 0}\textdollar\ gpg --sign-key KEYID}\\
\end{shaded}
\item 2c. Inviamola nuovamente al keyserver con appresso la nostra firma.\\
\begin{shaded}
{\color[cmyk]{0, 0, 0, 0}\textdollar\ gpg --send-key KEYID}\\
\end{shaded}
\end{itemize}

{\bfseries\centering Punto 3. Cifratura di un file decifrabile solamente da una determinata chiave.}\\

\begin{itemize}
 \item 3a. Otteniamo la chiave di cifratura dell'utente che poi dovrà effettivamente decifrare il nostro messaggio.\\
\begin{shaded}
{\color[cmyk]{0, 0, 0, 0}\textdollar\ gpg --search-keys 'nomeutente@example.com' --keyserver hkp://keys.fedoraproject.org} \\
\end{shaded}
\item 3b. Dopo aver selezionato la chiave corretta dalla lista, verificate che l'importazione è avvenuta correttamente:\\
\begin{shaded}
{\color[cmyk]{0, 0, 0, 0}\textdollar\ gpg --list-keys}
\end{shaded}
\item 3c. Infine, cifrate il messaggio:\\
\begin{shaded}
{\color[cmyk]{0, 0, 0, 0}\textdollar\ gpg --encrypt --recipient 'nomeutente@example.com' nomefile.txt}\\
\end{shaded}
\item 3d. Inviate via mail il file appena generato ('nomefile.txt.gpg') all'utente da voi designato, che potrà, quindi, decifrarlo:\\
\begin{shaded}
{\color[cmyk]{0, 0, 0, 0}\textdollar\ gpg --output nomefile.txt --decrypt nomefile.txt.gpg}
\end{shaded}
In questo caso non servirà importare nessuna chiave dal momento in cui il file di cui sopra è stato cifrato utilizzando la vostra chiave pubblica.\\
\end{itemize}

