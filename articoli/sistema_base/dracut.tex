\twocolumn[
\begin{center}
\title{\color[cmyk]{1, 0.57, 0, 0.38}{\Huge\bfseries Dracut \\}} % definisco il titolo dell'articolo
\author{\scriptsize Gabriele Trombini (mailga@fedoraonline.it)} % definisco l'autore e altre informazioni
\date{}
\end{center}
{\color[cmyk]{1, 0.46, 0, 0}\LARGE L'initramfs all'avvio di Fedora}\\
\maketitle
\normalsize
\doublespacing
\hfill
]
\definecolor{shadecolor}{cmyk}{0, 0, 0, 0.8}
\onehalfspacing
\lettrine[lines=1, loversize=0.1, lraise=0.1]{\color[cmyk]{0.5, 0, 1, 0}\bfseries C}{}ome sappiamo initrd e initramfs (per comodità diciamo che quest'ultimo è la versione attuale del precedente) non sono altro che dei file systems temporanei, che si collocano nella memoria, usati per caricare e montare il file system reale nel sistema.\\

Per compatibilità con un sistema generico, vengono inseriti al boot parecchi moduli di periferica e di filesystem ed il kernel, aumentando nelle dimensioni e nella quantità di hardware riconosciuto, impiega più tempo a caricarsi perciò si è reso necessario trovare soluzioni che mantengano un tempo ragionevole di start up.\\

A questo scopo è stata introdotta una fase iniziale di root file system in user space che si occupa di rilevare hardware, configurazioni, moduli prima di montare il file system root della macchina.\\

Difatti oggi la sequenza boot ordinariamente utilizzata è in due stages:
\begin{itemize}
 \item lancio e caricamento kernel con tutti i suoi moduli built-in ovvero i moduli integrati che sono indispensabili al corretto funzionamento di una macchina "base"
\item lancio di initramfs. Esso è di fatto un "micro sistema in ram" con le directory {\itshape bin - dev - etc- lib - tmp - var} etc... che provvede al caricamento non solo dei moduli atti al montaggio dei filesystem  e quindi del filesystem root, cosa fondamentale, ma anche al montaggio di altri moduli quali ad esempio gli alsa, quelli video (se necessario), con eventuali regole udev su determinati dispositivi, o altri moduli o performance che l'utente (esperto) decide di caricare subito con la modifica di {\itshape /etc/dracut.conf} o con l'inserimento di regole specifiche in {\itshape /etc/dracut.conf.d/}.
\end{itemize}

Dracut è quindi lo strumento che permette alla initial ramdisk di caricare solo quello che è necessario per l'avvio, creando una initramfs adatta al proprio sistema.\\

Tutto ciò avviene in automatico ma, qualora si avessero delle necessità particolari dovute ad installazioni successive di hardware o a moduli non interpretati dal sistema, è possibile utilizzare la linea di comando per effettuare le operazioni di creazione, variazione, verifica e manutenzione della inintramfs.\\

Il comando generico per poter creare un nuovo file system iniziale è:
\begin{shaded}
{\color[cmyk]{0, 0, 0, 0}\# dracut /boot/initramfs-\textdollar(uname -r).img \textdollar(uname -r)}\\
\end{shaded}

che viene utilizzato, ad esempio, dopo l'installazione dei driver grafici per poterli inserire già nel file system iniziale permettendo una corretta visualizzazione. \\

Come sempre in questi casi occorre sempre leggere il manuale dell'applicazione:

\begin{shaded}
{\color[cmyk]{0, 0, 0, 0}\textdollar\ man dracut}
\end{shaded}

all'interno del quale possiamo reperire le istruzioni e le opzioni per creare una immagine ramdisk iniziale integrando o escludendo moduli, filesystems, file di configurazione (compreso quello di deafult di dracut che è {\itshape /etc/dracut.conf}), partizioni e altro.\\

E' anche previsto un file di log, conseguente all'utilizzo di dracut, che trova posto all'interno della
directory {\itshape /var/log}, sotto il nome di {\itshape dracut.log}.

\begin{shaded}
{\color[cmyk]{0, 0, 0, 0}\textdollar\ cat  /var/log/dracut.log}
\end{shaded}

E' possibile inserire i parametri anche da linea di comando come descritto nella
pagina man:

\begin{shaded}
{\color[cmyk]{0, 0, 0, 0}\textdollar\ man dracut.kernel}
\end{shaded}

dove vediamo anche opzioni da passare al kernel all'avvio; conosciute ai più attenti perchè utilizzate anche nel grub.cfg di Fedora.\\

Può essere interessante guardare all'interno del file della initramfs utilizzando il tool {\itshape lsinitrd}
che ne permette l'ispezione:

\begin{shaded}
{\color[cmyk]{0, 0, 0, 0}\# lsinitrd /boot/initramfs-\textdollar(uname -r).img | more}
\end{shaded}

oppure verificare la presenza e contenuto di uno specifico file:

\begin{shaded}
{\color[cmyk]{0, 0, 0, 0}\# lsinitrd /boot/initramfs-\textdollar(uname -r).img run/initramfs/shutdown}
\end{shaded}

I file di configurazione di Dracut, in Fedora, devono essere posizionati all'interno della directory {\itshape /etc/dracut.conf.d/}, mentre il file da poter adattare alle proprie esigenze, commentandone e/o decommentandone le opzioni per meglio rifinire l'avvio, è {\itshape /etc/dracut.conf}.

\begin{shaded}
{\color[cmyk]{0, 0, 0, 0}\textdollar man dracut.conf}
\end{shaded}

Quanto inserito in {\itshape /etc/dracut.conf.d/} , in ordine numerico, va a sovrascrivere le impostazioni di {\itshape /etc/dracut.conf}, che invece è il file di prima lettura di dracut.\\

La particolarità di Dracut è data dal fatto che, proprio per adattare la singola macchina alle proprie necessità, si possa personalizzare la initramfs.\\

Utilizzando l'opzione {\itshape--include}, infatti, si possono inserire all'interno della initramfs i propri file di utilità particolare, ad esempio con il comando:

\begin{shaded}
{\color[cmyk]{0, 0, 0, 0}\# dracut --include /bin/cat / run/initramfs/usr/sbin initramfs-cat.img}
\end{shaded}
è possibile insterire il comando cat all'interno della directory {\itshape run/initramfs/usr/sbin} della initial ramdisk.\\

L'opzione {\itshape --install} consente di poter specificare diversi eseguibili che dovessimo ritenere utili all'avvio, rilevando anche le relative dipendenze:

\begin{shaded}
{\color[cmyk]{0, 0, 0, 0}\# dracut --install 'nmap ssh' initramfs-rete.img}
\end{shaded}

Vale la pena, anche, di ricordare che Dracut ha accesso ad una CLI, da dove si può cercare di fare un avvio manuale del sistema.\\

Togliendo  {\itshape rhgb} e {\itshape quiet} ed aggiungendo {\itshape rdshell} alle opzioni del kernel, si avrà a disposizione una shell con la quale cercare di eseguire il boot (per poi, magari, ricostruire la initial ramdisk una volta autenticati).\\

Tra i parametri utili per una analisi del sistema avremo a disposizione:

\begin{itemize}
 \item {\itshape rdshell}: come detto, inserendo questo comando si avrà accesso ad una shell di Dracut;
 \item {\itshape rdinitdebug}: per un livello di debug più approfondito, con stampa dei comandi;
 \item {\itshape rdbreak=[pre-udev | pre-mount | mount | pre-pivot |]}: aggiungendo questo parametro, la shell si fermerà prima di superare il punto dato come argomento;
 \item {\itshape rdudevinfo}: imposta udev ad un livello informativo massimo;
 \item {\itshape rdudevdebug}: imposta udev ad un livello di debug;
 \item {\itshape rdnetdebug}: verifica le connessioni network, l'output viene inserito in /tmp.
\end{itemize}

Questa breve panoramica non ha certo la pretesa di essere esaustiva ma quanto detto in questo articolo può concedere ampi spazi di approfondimento ai lettori.\\
